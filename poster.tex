%%%%%%%%%%%%%%%%%%%%%%%%%%%%%%%%%%%%%%
% LaTeX poster template
% Created by Nathaniel Johnston
% August 2009
% http://www.nathanieljohnston.com/2009/08/latex-poster-template/
%%%%%%%%%%%%%%%%%%%%%%%%%%%%%%%%%%%%%%

\documentclass[final]{beamer}
\usepackage[size=a0, scale=1.25]{beamerposter}
%\usepackage{atbegshi} 
%\AtBeginDocument{\AtBeginShipoutNext{\AtBeginShipoutDiscard}}

%-----------------------------------------------------------
% Define the column width and poster size
% To set effective sepwid, onecolwid and twocolwid values, first choose how many columns you want and how much separation you want between columns
% The separation I chose is 0.024 and I want 4 columns
% Then set onecolwid to be (1-(4+1)*0.024)/4 = 0.22
% Set twocolwid to be 2*onecolwid + sepwid = 0.464
%-----------------------------------------------------------

\newlength{\sepwid}
\newlength{\onecolwid}
\newlength{\twocolwid}
\newlength{\threecolwid}
\setlength{\paperwidth}{48in}
\setlength{\paperheight}{36in}
\setlength{\sepwid}{0.025\paperwidth}
\setlength{\onecolwid}{0.3\paperwidth}
\setlength{\twocolwid}{0.625\paperwidth}
\setlength{\threecolwid}{0.95\paperwidth}
\setlength{\topmargin}{-0.5in}
\usetheme{confposter}
\usepackage{exscale}

\usepackage{caption}
\captionsetup[figure]{labelformat=empty}

\usepackage{dsfont}
\usepackage{mathtools}
\usepackage{nicefrac}
\usepackage{tcolorbox}

\usepackage{algorithm}
\usepackage[noend]{algpseudocode}
\renewcommand\algorithmiccomment[1]{\quad// #1}

\newcommand*{\C}[1]{\mathcal{#1}}
\newcommand*{\R}[1]{\mathrm{#1}}
\newcommand*{\Z}[1]{\mathds{#1}}

\DeclareMathOperator*{\argmax}{arg\,max}
\DeclareMathOperator{\E}{\Z E}
\DeclareMathOperator{\p}{\Z P}

\newcommand*{\eq}[1]{\begin{align*}#1\end{align*}}
\newcommand*{\eqa}[1]{\begin{aligned}#1\end{aligned}}
\newcommand*{\eqat}[2]{\begin{alignat*}{#1}#2\end{alignat*}}
\newcommand*{\eqg}[1]{\begin{gather*}#1\end{gather*}}
\newcommand*{\lan}{\langle}
\newcommand*{\mrlap}{\mathrlap}
\newcommand*{\ran}{\rangle}


%-----------------------------------------------------------
% The next part fixes a problem with figure numbering. Thanks Nishan!
% When including a figure in your poster, be sure that the commands are typed in the following order:
% \begin{figure}
% \includegraphics[...]{...}
% \caption{...}
% \end{figure}
% That is, put the \caption after the \includegraphics
%-----------------------------------------------------------

\usecaptiontemplate{
\small
\structure{\insertcaptionname~\insertcaptionnumber:}
\insertcaption}

%-----------------------------------------------------------
% Define colours (see beamerthemeconfposter.sty to change these colour definitions)
%-----------------------------------------------------------

\definecolor{mygray}{RGB}{220,222,224}
\definecolor{mygreen}{RGB}{0, 165, 152}
\definecolor{myorange}{RGB}{255, 181, 21}
\definecolor{mydblue}{RGB}{0, 50, 97}
\definecolor{myblue}{RGB}{68, 137, 189}

\setbeamercolor{title in headline}{fg=mydblue}
\setbeamercolor{block title}{fg=mygreen,bg=white}
\setbeamercolor{block body}{fg=black,bg=white}
\setbeamercolor{block alerted title}{fg=black,bg=myorange}
\setbeamercolor{block alerted body}{fg=black,bg=white}
\setbeamercolor{item}{fg=mydblue}

\newtcolorbox{highlight}{colback=myblue!50}


%-----------------------------------------------------------
% Name and authors of poster/paper/research
%-----------------------------------------------------------

\title{Paper Title}
\author{First Author$^{*1}$, Second Author$^{*2}$, and Third Author$^3$}
\institute{$^*$~Equal contribution;~~~~$^1$~University of First Affiliation,~~$^2$~Second Affiliation Research Center,~~$^3$~The Third Affiliation Institute}

%-----------------------------------------------------------
% Start the poster itself
%-----------------------------------------------------------

\begin{document}
\addtobeamertemplate{headline}
{
\begin{tikzpicture}[remember picture,overlay] 
\node [shading=axis, rectangle, left color=mygray, right color=white, anchor=north, minimum width=\textwidth, minimum height=25em] (box) at (current page.north){};
\node [shift={(-22em,-10em)}] at (current page.north east) {\includegraphics[height=7em]{figures/rise_logo}};
\end{tikzpicture} 
}{}

\begin{frame}[t]
	\vspace{1.5em}
	\begin{columns}[t]								% the [t] option aligns the column's content at the top
		\begin{column}{\sepwid}\end{column}			% empty spacer column
		\begin{column}{\onecolwid}
			\begin{block}{Motivation}
				\vspace{.5em}
				\begin{highlight}
					{\bf Our motivation} is to find solutions that are
					\begin{itemize}
						\item correct;
						\item novel; and
						\item interesting.
					\end{itemize}
				\end{highlight}
			\end{block}
			\vspace{.5em}

			\begin{block}{Algorithm}
				\begin{center}\begin{minipage}{\columnwidth}
					\begin{algorithmic}
						\State Initialize
						\For{item in items}
							\If{condition holds}
								\State Do stuff
							\Else \Comment{nope}
								\State Don't, whatever
							\EndIf
						\EndFor
					\end{algorithmic}
				\end{minipage}\end{center}
			\end{block}
			\vspace{.5em}

			\begin{alertblock}{Alert}
				Stay alert!
			\end{alertblock}
			\vspace{.5em}
		\end{column}
		\begin{column}{\sepwid}\end{column}			% empty spacer column

		\begin{column}{\onecolwid}
		\end{column}
		\begin{column}{\sepwid}\end{column}			% empty spacer column

		\begin{column}{\onecolwid}
		\end{column}
		\begin{column}{\sepwid}\end{column}			% empty spacer column
	\end{columns}
\end{frame}
\end{document}
